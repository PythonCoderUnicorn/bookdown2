% Options for packages loaded elsewhere
\PassOptionsToPackage{unicode}{hyperref}
\PassOptionsToPackage{hyphens}{url}
%
\documentclass[
]{book}
\usepackage{amsmath,amssymb}
\usepackage{lmodern}
\usepackage{ifxetex,ifluatex}
\ifnum 0\ifxetex 1\fi\ifluatex 1\fi=0 % if pdftex
  \usepackage[T1]{fontenc}
  \usepackage[utf8]{inputenc}
  \usepackage{textcomp} % provide euro and other symbols
\else % if luatex or xetex
  \usepackage{unicode-math}
  \defaultfontfeatures{Scale=MatchLowercase}
  \defaultfontfeatures[\rmfamily]{Ligatures=TeX,Scale=1}
\fi
% Use upquote if available, for straight quotes in verbatim environments
\IfFileExists{upquote.sty}{\usepackage{upquote}}{}
\IfFileExists{microtype.sty}{% use microtype if available
  \usepackage[]{microtype}
  \UseMicrotypeSet[protrusion]{basicmath} % disable protrusion for tt fonts
}{}
\makeatletter
\@ifundefined{KOMAClassName}{% if non-KOMA class
  \IfFileExists{parskip.sty}{%
    \usepackage{parskip}
  }{% else
    \setlength{\parindent}{0pt}
    \setlength{\parskip}{6pt plus 2pt minus 1pt}}
}{% if KOMA class
  \KOMAoptions{parskip=half}}
\makeatother
\usepackage{xcolor}
\IfFileExists{xurl.sty}{\usepackage{xurl}}{} % add URL line breaks if available
\IfFileExists{bookmark.sty}{\usepackage{bookmark}}{\usepackage{hyperref}}
\hypersetup{
  pdftitle={RLadies Knowledge},
  pdfauthor={Zane Dax},
  hidelinks,
  pdfcreator={LaTeX via pandoc}}
\urlstyle{same} % disable monospaced font for URLs
\usepackage{color}
\usepackage{fancyvrb}
\newcommand{\VerbBar}{|}
\newcommand{\VERB}{\Verb[commandchars=\\\{\}]}
\DefineVerbatimEnvironment{Highlighting}{Verbatim}{commandchars=\\\{\}}
% Add ',fontsize=\small' for more characters per line
\usepackage{framed}
\definecolor{shadecolor}{RGB}{248,248,248}
\newenvironment{Shaded}{\begin{snugshade}}{\end{snugshade}}
\newcommand{\AlertTok}[1]{\textcolor[rgb]{0.94,0.16,0.16}{#1}}
\newcommand{\AnnotationTok}[1]{\textcolor[rgb]{0.56,0.35,0.01}{\textbf{\textit{#1}}}}
\newcommand{\AttributeTok}[1]{\textcolor[rgb]{0.77,0.63,0.00}{#1}}
\newcommand{\BaseNTok}[1]{\textcolor[rgb]{0.00,0.00,0.81}{#1}}
\newcommand{\BuiltInTok}[1]{#1}
\newcommand{\CharTok}[1]{\textcolor[rgb]{0.31,0.60,0.02}{#1}}
\newcommand{\CommentTok}[1]{\textcolor[rgb]{0.56,0.35,0.01}{\textit{#1}}}
\newcommand{\CommentVarTok}[1]{\textcolor[rgb]{0.56,0.35,0.01}{\textbf{\textit{#1}}}}
\newcommand{\ConstantTok}[1]{\textcolor[rgb]{0.00,0.00,0.00}{#1}}
\newcommand{\ControlFlowTok}[1]{\textcolor[rgb]{0.13,0.29,0.53}{\textbf{#1}}}
\newcommand{\DataTypeTok}[1]{\textcolor[rgb]{0.13,0.29,0.53}{#1}}
\newcommand{\DecValTok}[1]{\textcolor[rgb]{0.00,0.00,0.81}{#1}}
\newcommand{\DocumentationTok}[1]{\textcolor[rgb]{0.56,0.35,0.01}{\textbf{\textit{#1}}}}
\newcommand{\ErrorTok}[1]{\textcolor[rgb]{0.64,0.00,0.00}{\textbf{#1}}}
\newcommand{\ExtensionTok}[1]{#1}
\newcommand{\FloatTok}[1]{\textcolor[rgb]{0.00,0.00,0.81}{#1}}
\newcommand{\FunctionTok}[1]{\textcolor[rgb]{0.00,0.00,0.00}{#1}}
\newcommand{\ImportTok}[1]{#1}
\newcommand{\InformationTok}[1]{\textcolor[rgb]{0.56,0.35,0.01}{\textbf{\textit{#1}}}}
\newcommand{\KeywordTok}[1]{\textcolor[rgb]{0.13,0.29,0.53}{\textbf{#1}}}
\newcommand{\NormalTok}[1]{#1}
\newcommand{\OperatorTok}[1]{\textcolor[rgb]{0.81,0.36,0.00}{\textbf{#1}}}
\newcommand{\OtherTok}[1]{\textcolor[rgb]{0.56,0.35,0.01}{#1}}
\newcommand{\PreprocessorTok}[1]{\textcolor[rgb]{0.56,0.35,0.01}{\textit{#1}}}
\newcommand{\RegionMarkerTok}[1]{#1}
\newcommand{\SpecialCharTok}[1]{\textcolor[rgb]{0.00,0.00,0.00}{#1}}
\newcommand{\SpecialStringTok}[1]{\textcolor[rgb]{0.31,0.60,0.02}{#1}}
\newcommand{\StringTok}[1]{\textcolor[rgb]{0.31,0.60,0.02}{#1}}
\newcommand{\VariableTok}[1]{\textcolor[rgb]{0.00,0.00,0.00}{#1}}
\newcommand{\VerbatimStringTok}[1]{\textcolor[rgb]{0.31,0.60,0.02}{#1}}
\newcommand{\WarningTok}[1]{\textcolor[rgb]{0.56,0.35,0.01}{\textbf{\textit{#1}}}}
\usepackage{longtable,booktabs,array}
\usepackage{calc} % for calculating minipage widths
% Correct order of tables after \paragraph or \subparagraph
\usepackage{etoolbox}
\makeatletter
\patchcmd\longtable{\par}{\if@noskipsec\mbox{}\fi\par}{}{}
\makeatother
% Allow footnotes in longtable head/foot
\IfFileExists{footnotehyper.sty}{\usepackage{footnotehyper}}{\usepackage{footnote}}
\makesavenoteenv{longtable}
\usepackage{graphicx}
\makeatletter
\def\maxwidth{\ifdim\Gin@nat@width>\linewidth\linewidth\else\Gin@nat@width\fi}
\def\maxheight{\ifdim\Gin@nat@height>\textheight\textheight\else\Gin@nat@height\fi}
\makeatother
% Scale images if necessary, so that they will not overflow the page
% margins by default, and it is still possible to overwrite the defaults
% using explicit options in \includegraphics[width, height, ...]{}
\setkeys{Gin}{width=\maxwidth,height=\maxheight,keepaspectratio}
% Set default figure placement to htbp
\makeatletter
\def\fps@figure{htbp}
\makeatother
\setlength{\emergencystretch}{3em} % prevent overfull lines
\providecommand{\tightlist}{%
  \setlength{\itemsep}{0pt}\setlength{\parskip}{0pt}}
\setcounter{secnumdepth}{5}
\usepackage{booktabs}
\ifluatex
  \usepackage{selnolig}  % disable illegal ligatures
\fi
\usepackage[]{natbib}
\bibliographystyle{apalike}

\title{RLadies Knowledge}
\author{Zane Dax}
\date{2022-01-19}

\usepackage{amsthm}
\newtheorem{theorem}{Theorem}[chapter]
\newtheorem{lemma}{Lemma}[chapter]
\newtheorem{corollary}{Corollary}[chapter]
\newtheorem{proposition}{Proposition}[chapter]
\newtheorem{conjecture}{Conjecture}[chapter]
\theoremstyle{definition}
\newtheorem{definition}{Definition}[chapter]
\theoremstyle{definition}
\newtheorem{example}{Example}[chapter]
\theoremstyle{definition}
\newtheorem{exercise}{Exercise}[chapter]
\theoremstyle{definition}
\newtheorem{hypothesis}{Hypothesis}[chapter]
\theoremstyle{remark}
\newtheorem*{remark}{Remark}
\newtheorem*{solution}{Solution}
\begin{document}
\maketitle

{
\setcounter{tocdepth}{1}
\tableofcontents
}
\hypertarget{about}{%
\chapter{About}\label{about}}

This is a book project at trying to condense various \textbf{RLadies} meetings, videos, lectures and talks into a single book. This goal is super ambitious and one that will not be complete.

\hypertarget{usage}{%
\section{Usage}\label{usage}}

Each book chapter are topics covered by RLadies chapters, and their code that is from their github repository. The idea is to have each chapter dedicated to a topic despite which RLadies chapter presented it.

\hypertarget{rladies-book}{%
\section{RLadies book}\label{rladies-book}}

This book is an idea to have as much of RLadies knowledge that exists in Github repos, meetups and YouTube videos to be in one readable format. Due to the global reach of R chapters there will be redundancy. This book will focus on English presentations (for now) until someone else comes along to add to this project.

It is the hope this volume of knowledge is read and shared, along with additions to keep up with the copious RLadies meetups.

\hypertarget{hello-rladies}{%
\chapter{Hello RLadies}\label{hello-rladies}}

All chapters of \textbf{RLadies} are condensed under RLadies Global and each chapter will try to showcase the topic by as many RLadies chapters as possible without redundancy.

\hypertarget{start-here}{%
\section{StaRt here}\label{start-here}}

This section is about starting in RStudio using R. This starter code is from \textbf{RLadies Freiburg} repo. Starting out with using base R which entails the use of the \texttt{\$} dollar sign, which is used to grab a specific column from a dataset.

The dataset you will use is from a library called \emph{datasets} which is a R package that contains various datasets to be used for your learning needs. Using either a simple .R file or Rmd file with a \texttt{\{r\}} code chunk, enter the code below for basic descriptive statistics.

\begin{Shaded}
\begin{Highlighting}[]
\CommentTok{\#{-}{-}{-} step one is to install the library package}
\CommentTok{\# install.packages(\textquotesingle{}datasets\textquotesingle{})    \# comment out to run}
\FunctionTok{library}\NormalTok{(datasets)                 }\CommentTok{\# load library}

\CommentTok{\# the specific dataset we want is called ChickWeight}
\FunctionTok{data}\NormalTok{(}\StringTok{"ChickWeight"}\NormalTok{)               }\CommentTok{\# function to pull out specific dataset}

\FunctionTok{head}\NormalTok{(ChickWeight)                 }\CommentTok{\# see first 6 rows of dataset}
\CommentTok{\#\textgreater{}   weight Time Chick Diet}
\CommentTok{\#\textgreater{} 1     42    0     1    1}
\CommentTok{\#\textgreater{} 2     51    2     1    1}
\CommentTok{\#\textgreater{} 3     59    4     1    1}
\CommentTok{\#\textgreater{} 4     64    6     1    1}
\CommentTok{\#\textgreater{} 5     76    8     1    1}
\CommentTok{\#\textgreater{} 6     93   10     1    1}
\end{Highlighting}
\end{Shaded}

\hypertarget{descriptive-stats}{%
\section{Descriptive Stats}\label{descriptive-stats}}

For simple descriptive statistics in R is by using the base R functions like \texttt{mean()}, \texttt{std()} and \texttt{max()}, among others. For this ChickWeight dataset, we want to find the mean weight.

\begin{Shaded}
\begin{Highlighting}[]
\CommentTok{\# find the mean weight. }
\CommentTok{\# need to use the $ to pull the data from column \textquotesingle{}weight\textquotesingle{}}

\CommentTok{\# {-} method 1:}
\CommentTok{\# use the function mean with our dataset$column}
\FunctionTok{mean}\NormalTok{(ChickWeight}\SpecialCharTok{$}\NormalTok{weight)  }
\CommentTok{\#\textgreater{} [1] 121.8183}

\CommentTok{\# {-} method 2:}
\CommentTok{\# make a variable to store our column data}
\NormalTok{chik\_wt }\OtherTok{=}\NormalTok{ ChickWeight}\SpecialCharTok{$}\NormalTok{weight}

\NormalTok{avg\_chick\_wt }\OtherTok{=} \FunctionTok{mean}\NormalTok{(chik\_wt) }\CommentTok{\# save the mean chick weight as variable}
\NormalTok{avg\_chick\_wt}
\CommentTok{\#\textgreater{} [1] 121.8183}
\end{Highlighting}
\end{Shaded}

Now we want to create a new variable for our deviation from the mean weight.

\begin{Shaded}
\begin{Highlighting}[]
\CommentTok{\# create new column.}
\CommentTok{\# dataset$NEW\_COLUMN\_NAME }

\NormalTok{ChickWeight}\SpecialCharTok{$}\NormalTok{Deviation }\OtherTok{=}\NormalTok{  ChickWeight}\SpecialCharTok{$}\NormalTok{weight }\SpecialCharTok{{-}}\NormalTok{ avg\_chick\_wt}
\NormalTok{ChickWeight}\SpecialCharTok{$}\NormalTok{Deviation[}\DecValTok{1}\SpecialCharTok{:}\DecValTok{10}\NormalTok{]}
\CommentTok{\#\textgreater{}  [1] {-}79.818339 {-}70.818339 {-}62.818339 {-}57.818339 {-}45.818339}
\CommentTok{\#\textgreater{}  [6] {-}28.818339 {-}15.818339   3.181661  27.181661  49.181661}
\end{Highlighting}
\end{Shaded}

\hypertarget{for-loop}{%
\section{For Loop}\label{for-loop}}

We can create a for loop to make our ChickWeight data into 2 categories by using a \texttt{ifelse} statement to make categories: `Normal' and `Large'.

\begin{Shaded}
\begin{Highlighting}[]
\CommentTok{\#  for loop to make categories Normal and Large based on weight}
\ControlFlowTok{for}\NormalTok{ (x }\ControlFlowTok{in} \DecValTok{1}\SpecialCharTok{:}\FunctionTok{nrow}\NormalTok{(ChickWeight)) \{}
\NormalTok{  ChickWeight}\SpecialCharTok{$}\NormalTok{Large }\OtherTok{=} \FunctionTok{ifelse}\NormalTok{(ChickWeight}\SpecialCharTok{$}\NormalTok{Deviation }\SpecialCharTok{\textless{}} \DecValTok{100}\NormalTok{, }\StringTok{"Normal"}\NormalTok{, }\StringTok{"Large"}\NormalTok{)}
\NormalTok{\}}
\end{Highlighting}
\end{Shaded}

\hypertarget{data-visual}{%
\section{Data Visual}\label{data-visual}}

We can use the base R plotting function \texttt{plot()} to plot our data.

\begin{Shaded}
\begin{Highlighting}[]
\FunctionTok{plot}\NormalTok{(ChickWeight}\SpecialCharTok{$}\NormalTok{weight)}
\end{Highlighting}
\end{Shaded}

\includegraphics{01-intro_files/figure-latex/unnamed-chunk-5-1.pdf}

\begin{Shaded}
\begin{Highlighting}[]
\FunctionTok{plot}\NormalTok{(ChickWeight}\SpecialCharTok{$}\NormalTok{Diet, ChickWeight}\SpecialCharTok{$}\NormalTok{weight)}
\end{Highlighting}
\end{Shaded}

\includegraphics{01-intro_files/figure-latex/unnamed-chunk-6-1.pdf}

\begin{Shaded}
\begin{Highlighting}[]
\CommentTok{\# histogram}
\FunctionTok{hist}\NormalTok{(ChickWeight}\SpecialCharTok{$}\NormalTok{Deviation)}
\end{Highlighting}
\end{Shaded}

\includegraphics{01-intro_files/figure-latex/unnamed-chunk-7-1.pdf}

\hypertarget{tidyverse}{%
\chapter{Tidyverse}\label{tidyverse}}

The tidyverse has its own book called \emph{R for Data Science} by Hadley Wickham \& Garrett Grolemund (available online). This section will discuss a few functions out of the library that are very useful to know.

\hypertarget{tidy-data}{%
\section{Tidy Data}\label{tidy-data}}

Data Transformation and Introduction to \textbf{Tidyverse} library. This section is by \textbf{RLadies Freiburg} Divya Seernani.
Tidy data is data that has data or values for each column (variable) and each row (observation). The paradigm:
\texttt{import} -\textgreater{} \texttt{tidy} -\textgreater{} {[}Transform \textless-\textgreater{} Visualize \textless-\textgreater{} Model{]} -\textgreater{} Communicate

\hypertarget{dplyr-library}{%
\subsection{dplyr library}\label{dplyr-library}}

Inside the \textbf{tidyverse} library is the dplyr library which is helpful when creating new variables, renaming or reordering observations, selecting specific observations and calculating summary statistics among many other functions.

The dataset for this section will be the Indian Census (2011), dataset is available \href{https://github.com/nishusharma1608/India-Census-2011-Analysis/blob/master/india-districts-census-2011.csv}{here}, \emph{Remember to click on `raw' data format first to load in data, you should see white background and black text CSV data, copy the url link in order to read in the data}.

The tidyverse library brings along other libraries and when you load tidyverse you will a bunch of warnings in the console, these are no concern to you, basically telling you that one library hides another for function use. To turn off warnings in your Rmd file, inside the code chunk \{r\} set warning to false \texttt{\{r,\ warning=\ FALSE,\ message=\ FALSE\}}.
s

\begin{Shaded}
\begin{Highlighting}[]
\CommentTok{\# step 1 {-} install tidyverse library}
\CommentTok{\# install.packages(\textquotesingle{}tidyverse\textquotesingle{})  \# comment{-}out to run }
\FunctionTok{library}\NormalTok{(tidyverse)}

\CommentTok{\# read in the data}
\NormalTok{india\_census }\OtherTok{=} \FunctionTok{read.csv}\NormalTok{(}\StringTok{\textquotesingle{}https://raw.githubusercontent.com/nishusharma1608/India{-}Census{-}2011{-}Analysis/master/india{-}districts{-}census{-}2011.csv\textquotesingle{}}\NormalTok{)}
\end{Highlighting}
\end{Shaded}

\hypertarget{section-2}{%
\section{section 2}\label{section-2}}

\hypertarget{parts}{%
\chapter{Parts}\label{parts}}

You can add parts to organize one or more book chapters together. Parts can be inserted at the top of an .Rmd file, before the first-level chapter heading in that same file.

Add a numbered part: \texttt{\#\ (PART)\ Act\ one\ \{-\}} (followed by \texttt{\#\ A\ chapter})

Add an unnumbered part: \texttt{\#\ (PART\textbackslash{}*)\ Act\ one\ \{-\}} (followed by \texttt{\#\ A\ chapter})

Add an appendix as a special kind of un-numbered part: \texttt{\#\ (APPENDIX)\ Other\ stuff\ \{-\}} (followed by \texttt{\#\ A\ chapter}). Chapters in an appendix are prepended with letters instead of numbers.

\hypertarget{footnotes-and-citations}{%
\chapter{Footnotes and citations}\label{footnotes-and-citations}}

\hypertarget{footnotes}{%
\section{Footnotes}\label{footnotes}}

Footnotes are put inside the square brackets after a caret \texttt{\^{}{[}{]}}. Like this one \footnote{This is a footnote.}.

\hypertarget{citations}{%
\section{Citations}\label{citations}}

Reference items in your bibliography file(s) using \texttt{@key}.

For example, we are using the \textbf{bookdown} package \citep{R-bookdown} (check out the last code chunk in index.Rmd to see how this citation key was added) in this sample book, which was built on top of R Markdown and \textbf{knitr} \citep{xie2015} (this citation was added manually in an external file book.bib).
Note that the \texttt{.bib} files need to be listed in the index.Rmd with the YAML \texttt{bibliography} key.

The \texttt{bs4\_book} theme makes footnotes appear inline when you click on them. In this example book, we added \texttt{csl:\ chicago-fullnote-bibliography.csl} to the \texttt{index.Rmd} YAML, and include the \texttt{.csl} file. To download a new style, we recommend: \url{https://www.zotero.org/styles/}

The RStudio Visual Markdown Editor can also make it easier to insert citations: \url{https://rstudio.github.io/visual-markdown-editing/\#/citations}

\hypertarget{blocks}{%
\chapter{Blocks}\label{blocks}}

\hypertarget{equations}{%
\section{Equations}\label{equations}}

Here is an equation.

\begin{equation} 
  f\left(k\right) = \binom{n}{k} p^k\left(1-p\right)^{n-k}
  \label{eq:binom}
\end{equation}

You may refer to using \texttt{\textbackslash{}@ref(eq:binom)}, like see Equation \eqref{eq:binom}.

\hypertarget{theorems-and-proofs}{%
\section{Theorems and proofs}\label{theorems-and-proofs}}

Labeled theorems can be referenced in text using \texttt{\textbackslash{}@ref(thm:tri)}, for example, check out this smart theorem \ref{thm:tri}.

\begin{theorem}
\protect\hypertarget{thm:tri}{}\label{thm:tri}For a right triangle, if \(c\) denotes the \emph{length} of the hypotenuse
and \(a\) and \(b\) denote the lengths of the \textbf{other} two sides, we have
\[a^2 + b^2 = c^2\]
\end{theorem}

Read more here \url{https://bookdown.org/yihui/bookdown/markdown-extensions-by-bookdown.html}.

\hypertarget{callout-blocks}{%
\section{Callout blocks}\label{callout-blocks}}

The \texttt{bs4\_book} theme also includes special callout blocks, like this \texttt{.rmdnote}.

You can use \textbf{markdown} inside a block.

\begin{Shaded}
\begin{Highlighting}[]
\FunctionTok{head}\NormalTok{(beaver1, }\AttributeTok{n =} \DecValTok{5}\NormalTok{)}
\CommentTok{\#\textgreater{}   day time  temp activ}
\CommentTok{\#\textgreater{} 1 346  840 36.33     0}
\CommentTok{\#\textgreater{} 2 346  850 36.34     0}
\CommentTok{\#\textgreater{} 3 346  900 36.35     0}
\CommentTok{\#\textgreater{} 4 346  910 36.42     0}
\CommentTok{\#\textgreater{} 5 346  920 36.55     0}
\end{Highlighting}
\end{Shaded}

It is up to the user to define the appearance of these blocks for LaTeX output.

You may also use: \texttt{.rmdcaution}, \texttt{.rmdimportant}, \texttt{.rmdtip}, or \texttt{.rmdwarning} as the block name.

The R Markdown Cookbook provides more help on how to use custom blocks to design your own callouts: \url{https://bookdown.org/yihui/rmarkdown-cookbook/custom-blocks.html}

\hypertarget{sharing-your-book}{%
\chapter{Sharing your book}\label{sharing-your-book}}

\hypertarget{publishing}{%
\section{Publishing}\label{publishing}}

HTML books can be published online, see: \url{https://bookdown.org/yihui/bookdown/publishing.html}

\hypertarget{pages}{%
\section{404 pages}\label{pages}}

By default, users will be directed to a 404 page if they try to access a webpage that cannot be found. If you'd like to customize your 404 page instead of using the default, you may add either a \texttt{\_404.Rmd} or \texttt{\_404.md} file to your project root and use code and/or Markdown syntax.

\hypertarget{metadata-for-sharing}{%
\section{Metadata for sharing}\label{metadata-for-sharing}}

Bookdown HTML books will provide HTML metadata for social sharing on platforms like Twitter, Facebook, and LinkedIn, using information you provide in the \texttt{index.Rmd} YAML. To setup, set the \texttt{url} for your book and the path to your \texttt{cover-image} file. Your book's \texttt{title} and \texttt{description} are also used.

This \texttt{bs4\_book} provides enhanced metadata for social sharing, so that each chapter shared will have a unique description, auto-generated based on the content.

Specify your book's source repository on GitHub as the \texttt{repo} in the \texttt{\_output.yml} file, which allows users to view each chapter's source file or suggest an edit. Read more about the features of this output format here:

\url{https://pkgs.rstudio.com/bookdown/reference/bs4_book.html}

Or use:

\begin{Shaded}
\begin{Highlighting}[]
\NormalTok{?bookdown}\SpecialCharTok{::}\NormalTok{bs4\_book}
\end{Highlighting}
\end{Shaded}


  \bibliography{book.bib,packages.bib}

\end{document}
